% \documentclass[14pt]{beamer}
\documentclass{beamer}
% \documentclass[aspectratio=169]{beamer}

\usetheme{Copenhagen}
% \usetheme{Boadilla}
% \usecolortheme{beaver}
\setbeamercolor{alerted text}{fg=orange}
\setbeamercolor{background canvas}{bg=white}
\setbeamercolor{block body alerted}{bg=normal text.bg!90!black}
\setbeamercolor{block body}{bg=normal text.bg!90!black}
\setbeamercolor{block body example}{bg=normal text.bg!90!black}
\setbeamercolor{block title alerted}{use={normal text,alerted text},fg=alerted text.fg!75!normal text.fg,bg=normal text.bg!75!black}
% \setbeamercolor{block title}{bg=blue}
\setbeamercolor{block title example}{use={normal text,example text},fg=example text.fg!75!normal text.fg,bg=normal text.bg!75!black}
\setbeamercolor{fine separation line}{}
\setbeamercolor{frametitle}{fg=white}
\setbeamercolor{item projected}{fg=white}
\setbeamercolor{normal text}{bg=white,fg=black}
\setbeamercolor{palette sidebar primary}{use=normal text,fg=normal text.fg}
\setbeamercolor{palette sidebar quaternary}{use=structure,fg=structure.fg}
\setbeamercolor{palette sidebar secondary}{use=structure,fg=structure.fg}
\setbeamercolor{palette sidebar tertiary}{use=normal text,fg=normal text.fg}
\setbeamercolor{section in sidebar}{fg=brown}
\setbeamercolor{section in sidebar shaded}{fg=grey}
\setbeamercolor{separation line}{}
\setbeamercolor{sidebar}{bg=red}
\setbeamercolor{sidebar}{parent=palette primary}

\setbeamercolor{structure}{bg=black, fg=white!30!blue!70!green}

\setbeamercolor{subsection in sidebar}{fg=brown}
\setbeamercolor{subsection in sidebar shaded}{fg=grey}
\setbeamercolor{title}{fg=white}
\setbeamercolor{titlelike}{fg=white}

\setbeamerfont{bibliography item}{size=\tiny}
\setbeamerfont{bibliography entry author}{size=\tiny}
\setbeamerfont{bibliography entry title}{size=\tiny}
\setbeamerfont{bibliography entry location}{size=\tiny}
\setbeamerfont{bibliography entry note}{size=\tiny}

% Szép kék
% \setbeamercolor{structure}{bg=black, fg=white!10!green!40!blue}

\frenchspacing

% Language packages
\usepackage[utf8]{inputenc}
\usepackage[T1]{fontenc}
\usepackage[magyar]{babel}

% AMS
\usepackage{amssymb,amsmath}

% Graphic packages
\usepackage{graphicx}

% Syntax highlighting
\usepackage{listings}

\usepackage{tikz}

\title[Adatbázis rendszerek MSc gyakorlat]{}
\author[Nagy Dániel Zoltán JJ181J]{}

% ==============
\begin{document}
% ==============

% --------------------
\begin{frame}[fragile]
\frametitle{1.7 feladat}
Adott az alábbi tábla:\\
KÖNYV (isbn C(20) PK, cim C(40), targy C(30), ar INT)


Adja meg az alábbi műveletek relációs algebrai alakját.


- a könyvek darabszáma
\begin{align*}
\Gamma^{\text{COUNT(*)}}\text{(KÖNYV)}
\end{align*}
- a könyvek átlagára
\begin{align*}
\Gamma^{\text{AVG(ár)}}\text{(KÖNYV)}
\end{align*}
- a legolcsóbb könyv ára
\begin{align*}
\Gamma^{\text{MIN(ár)}}\text{(KÖNYV)}
\end{align*}
- az ‘AB’ kategóriájú könyvek darabszáma
\begin{align*}
\Gamma^{\text{COUNT(*)}}(\sigma_{\text{tárgy = 'AB'}}\text{(KÖNYV))}
\end{align*}

\end{frame}

% --------------------
\begin{frame}[fragile]
\frametitle{1.7 feladat}

- a legdrágább AB kategóriájú könyv ára
\begin{align*}
\Gamma^{\text{MAX(ár)}}(\sigma_{\text{tárgy = 'AB'}}\text{(KÖNYV))}
\end{align*}
- az átlagárnál drágább könyvek címei
\begin{align*}
\Pi_{\text{cím}}(\sigma_{\text{ár} > \Gamma^{\text{AVG(ár)}}}\text{(KÖNYV))}
\end{align*}
- az átlagárnál drágább könyvek darabszáma
\begin{align*}
\Gamma^{\text{COUNT(*)}}(\sigma_{\text{ár} > \Gamma^{\text{AVG(ár)}}}\text{(KÖNYV))}
\end{align*}


\end{frame}

\end{document}
